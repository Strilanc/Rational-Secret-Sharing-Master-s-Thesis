\documentclass{article}
\usepackage{textcomp}
\begin{document}

\section{Introduction}

Blurb about secret sharing.

Blurb about game theory and rationality.

\section{Results}

New fancy algorithm in the syncrhonous domain.

Exploring the asynchronous domain.

\section{Rational Secret Sharing}
\subsection{Secret Sharing}
Dividing a secret into parts, such that some minimum number of parts are needed to reconstruct the original secrent.

\subsection{Rational Secret Sharing}
The secret shares have been given to rational players who prefer to know the secret, then prefer to minimize the number of other players who know the secret.

The protocol has to work in the present or rational players.

If the protocol has a fixed number of rounds, a rational player will defect on the last round because that is a dominant strategy.
		
Thus the number of rounds must be unknown to the players. Ideally each round should have the maximum possible chance of being the last round without incentivizing players to defect.

\subsection{Goals}
\begin{itemize}
	\item Minimize the impact of coalitions (of up to m-1 players, since larger coalitions can independently derive the secret).
	\item Maximize the number of participants that learn the secret.
	\item Prevent denial of service.
	\item Prevent defection.
\end{itemize}

\section{Importance}
tod This seems like a practical problem. Any scenario with shares distributed to parties that don't trust each other.

Suppose a will is encrypted with a key distributed to the various heirs. If the will can't be decrypted the fortune is distributed evenly. If a player can prevent others from seeing the will conditionally on that player getting less than a uniform share, they have an incentive to learn first then conditionally DOS.

Suppose a valuable database is protected by a key distributed amonst various players. The value of information may be higher if it is known to a single party, so they have an incentive to prevent others from learning the secret.

\section{Common Assumptions In Previous Work}
\subsection{Cryptographic Primitives}
\subsection{Synchronous Broadcast}
Existing algorithms (check!) all depend on synchronous broadcast. In other words, they assume players can communicate such that:
\begin{itemize}
\item Obliviousness.

	No player can receive a round's messages before sending their own round message. This prevents waiting to receive all messages and confirming it is not the last round before sending your own message, resulting in a deadlock due to all players waiting on each other.

\item Reliability.

	Messages are never mangled or lost. Receiving a message is equivalent to a player sending that message. This prevents confusing non-participating players with unlucky players.

\item Cross-consistency.

	Players can't send differing messages to different players (at least on the broadcast channel). This prevents byzantine complications in some protocols. (How often is this relied upon? Is it necessary?)
\end{itemize}

I will cover some methods of removing syncrhonous broadcast assumptions.

\section{Previous Work}
\subsection{Rational secret sharing and multiparty computation: extended abstract}
Proves there is no finite bound on the number of rounds by deleting weakly dominated strategies.

Introduces a 3 of 3 protocol with finite expected number of rounds.

Extends to an m of n protocol (n > 2, n >= m).

The protocol has the issuer giving out new shares of the secret each round. Requiring the presence of the issuer defeats the purpose of secret sharing, because the shares become unnecessary. Just have the issuer give out the secret when enough players indicate they want the secret revealed.

Assumptions about utility functions.

Uses synchronous broadcast.

\subsection{Collusion Free Protocol for Rational Secret Sharing}

Uses synchronous broadcast.

\subsection{Games for Exchanging Information}

Short list long list. Short player knows definitive round, long player has value but doesn't know which. O(t) information shared!

\subsection{Fairness with an Honest Minority and a Rational Majority}

\subsection{Rational Secret Sharing with Repeated Games}
Poor quality paper.
Covers asynchronous case.
But assumes the dealer is present for all rounds, allowing ack optimization.

\subsection{The deterministic protocol for rational secret sharing}

\subsection{How to Share a Secret}
Shamir's (and who?) secret sharing protocol uses polynomials over finite fields. For a secret requiring M shares to reveal, the sharer generates a random M-1 degree polynomial with the secret stored in the y value at x=0. The ith share is the value of the polynomial at x=i. Polynomial interpolation of an (M-1)th degree polynomial requires M points, and fewer points allow any value at 0 (clearer...).

This scheme has several benefits:
\begin{itemize}
	\item Secure. Given m-1 shares, every y value at x=0 is equally likely. (Why?)
	\item Extensible. New shares can be introduced without modifying or weakening the old shares. Players don't even need to know how many shares there are in total or who has what shares.
	\item Compact. Optimal amount of information per share. Each player gets an amount of information equal to the information required to store the secret. (This is optimal because if a share had less information than the secret, m-1 shares would be able to know something about the secret based on fewer possible remaining shares than secrets.)
	\item and...?
\end{itemize}

\subsection{paper using homomorphic encryption to emulate ack-ing\ldots}
Emulating a trusted intermediary with homomorphic encryption. Works for 2 players.

Very expensive. Implementation details very complicated? Relies on multi party computation (is secret sharing supposed to enable more efficient multi party computation?).

\section{Types of Players and Coalitions}

\begin{itemize}
  \item Honest
  \subitem Tries to follow the protocol exactly.
  \subitem Doesn't participate in coalitions.
  \subitem M of N with H honest players is equivalent to (M-H) of (N-H) with 0 honest players.
  \item Irrational
  \subitem Wants to prevent others from learning the secret. Doesn't care if they themselves learn the secret.
  \subitem Irrational coalitions don't care if their members learn the secret.
  \item Rational
  \subitem Wants to learn the secret above all else and, secondarily, wants to prevent others from learning it.
  \subitem Rational coalitions of size C are similar to a rational player with C puppets.
  \subitem Rational coalitions of size M or more are equivalent to irrational coalitions. 
\end{itemize}

\begin{itemize}
  \item Assume irrational coalitions don't attempt to deny their own members access to the secret, in order to allow them to be strong. 
  \item Mixed rational/irrational coalitions are not equivalent to pure rational or pure irrational coalitions.
  \item If irrational opponents can prove to others when the last round occurs, before it does, the rational players will not participate.
  \item Irrational coalitions with at least M members know when the last round occurs.
  \item Irrational coalitions have two strategies against our algorithm: target m-1 or target m-2. Targeting m-1 requires fewer sacrifices but is not cooperative with other irrationals and only prevents m-1 non-irrationals from learning the secret. Targeting m-2 is cooperative with other irrationals but only works if there are n-m+1 irrationals.
  \item Rational coalitions have the omit-last-send-when-you-can strategy.
  \item 
\end{itemize}
Opponents are organized into coalitions, which freely share information amonst themselves in order to further their goals. Simple results:

What about coalitions of half rational half irrational players? Do they always categorize as rational or irrational? NO: The rationals won't sacrifice themselves for the good of the coalition, because they will then be excluded.

- If it is possible to prove when the last round occurs without revealing the secret, the irrational opponents can do so in order to cause the rational opponents to stop playing.
- If an irrational coalition has more than m members, they know when the last round occurs.
- If the irrational opponents know there are fewer than n-m non-irrational players, they will all stop playing so the secret will not be learned.
- m rational or honest players will learn the secret if there is synchronous broadcast.
- m honest players will learn the secret if messages are verifiable.
- coalitions are significantly less powerful if they don't have the n-m majority. Not clear how multiple coalitions affect the result.

\section{Emulating Simultaneous Broadcast}

We can emulate simultaneous broadcast with some losses. Strategies:

\subsection{Use finite sync broad}
If we have m sync, choose m at random each round to send. Everyone playing learns the secret. Defeats rational coalitions. Irrational coalitions can sacrifice two members per round to strengthen. How much stronger are they??

If we have m-1 sync, choose m-1 at random each round to send. Everyone except last round senders learn the secret. Defeats rational coalitions. Irrational coalitions. Irrational coalitions only have to sacrifice one member per round they are chosen. Again, how much stronger?
Other m-1 strategy: send 1 then m-1. Rational coalitions with c=m-1 a 1 - c/n chance of winning. Irrational coalitions have to sacrifice two per round.

If we have m-2 sync: send 1 then m-2. Rational coalitions with c=m-1 have a 1 - c/n chance of winning. Irrational coalitions have to sacrifice 1 per round.
Alt: send 2 then m-2. Rational coalitions with c = m-1 have a 1 - (c choose 2)/(n choose 2) chance of winning. Irrationals sacrifice 2 per round.

Targeting m-1 allows single irrational sacrifices but costs the senders learning the secret and an increased viability for rational coalitions.
Targeting m requires double irrational sacrifices, unless k=1,  

\section{Optimal Algorithm for Synchronous Broadcast}
\subsection{Dealer Protocol}
\begin{itemize}
	\item Generate a commitment to the secret and a nonce
	\item Generate a public/private key pair for each player
	\item Pick a poisson-distributed round to be the final round
	\item Compute secret shares based on any secret sharing scheme (e.g. shamir)
	\item Compute a masked share for each player 
	\subitem The true share will be revealed on the target round
	\subitem maskedShare = PrivateEncrypt(round + nonce) xor actualShare
	\item Send each player their private key
	\item Broadcast to every player the nonce, commitment, all masked shares, and all public keys
\end{itemize}
\subsection{Player Protocol}
\begin{itemize}
	\item Broadcast this round's potential share to all players
	\subitem potentialShare = PrivateEncrypt(round + nonce)
	\item Verify potential shares received from all players
	\subitem PublicDecrypt(received) == round + nonce
	\subitem Exclude players from future rounds if they send nothing or send invalid potential shares
	\item Unmask the potential shares to compute the potential secret
	\subitem unmaskedShare(p) = receivedValue(p) xor maskedShare(p)
	\subitem potentialSecret = combine(unmaskedShares)
	\item Start next round if the potential secret doesn't match the commitment
	\item Otherwise the secret has been learned
\end{itemize}

\subsection{Properties}
\begin{itemize}
	\item Players can't know if a round is the last round before sending their share for that round.
	\subitem Sends are synchronized and other shares must be received in order to create the potential secret in order to compare against the commitment.
	\item Perfect resilience against coalitions up to m-1 players, the best possible.
	\item Works for 2 of 2 (n>=m, m>=2).
	\item O(N) shared values, O(N) private values
	\item O(N) cost per round with verification (O(M) if verification skipped)
	\item Cost per round has a much smaller constant than using homomorphic encryption.
	\item Optimal resilience against chaotic players. As long as m players follow the protocol, the secret is learned.
\end{itemize}

\subsection{Correctness}

If a player defects before the last round, they do not learn the secret.

If a player defects on the last round, others may be prevented from learning the secret.

A player can't determine if a round is the final round before deciding to defect. They need other players' round shares to determine if it's the last round.

A rational player will not defect, because they prefer to learn the secret and won't chance defecting on the incorrect round.

\section{Non-Synchronous Broadcast}

All previous papers have relied upon the assumption of synchronous broadcast. Essentially, the ability to guarantee multiple players send messages before receiving messages sent by other players. This assumption is not very practical.

Losing the synchronous broadcast assumption has costs. Fewer players will learn the secret, and coalitions become more powerful. Examples with proof below.

\subsection{If all players are rational then at least one will not learn the secret}
\begin{itemize}
  \item Let all players be rational
  \item Assume there exists a first round R where all players know the secret, $R >= M-1$
  \subitem Let $S$ be the sender of round R's message
  \subitem $S$ knows the secret in round R
  \subsubitem By definition of R
  \subitem $S$ learned no new information during round R
  \subsubitem $S$ only sent information
  \subitem $S$ knew the secret in round R-1
  \subsubitem $S$ had the same information then
  \subitem $S$ rationally prefers fewer players to learn the secret 
  \subsubitem By definition
  \subitem For $S$, not sending the message weakly dominates sending the message
  \subsubitem $S$ knows the secret either way
  \subsubitem $S$ knows sending messages may reveal the secret to more players
  \subitem $S$ will not send a message in round R
  \subitem $S$ sent a message in round R
  \subitem Contradiction
  \item There is no first round where all players learn the secret
  \subitem By contradiction
  \item At least one player doesn't learn the secret
\end{itemize}

Note that we can achieve exactly one player not learning the secret in the absence of coalitions, meaning the bound is tight. The algorithm will be explored later.

\subsection{Coalitions have a chance of learning the secret earlier}
\begin{itemize}
  \item Let $R$ be the first round where any player learns the secret legitimately (assuming no collusion), $R >= M-1$
  \item Let $C$ be a set of players in a rational coalition, $1 < |C| < M$
  \item Let $L$ be the set of players who learn the secret in round $R$, $L \not = \emptyset$.
  \item Let $S$ be the sender of round R's message
  \item $S \not \in L$
    \subitem S learned no new information during round R
  \item $R >= |C|$
    \subitem By $R >= M-1$, $|C| < M$
  \item If $L \cap S \not = \emptyset and S \in C$ then in round R-1 the coalition knows the information necessary to reconstruct the secret.
  \item $P(L \cap S \not = \emptyset and S \in C) >= \frac{|C|*(|C|-1)}{N*(N-1)}$
  	\subitem $P(S \in C) = \frac{|C|}{N}$
    \subitem If $|L| = 1$ then $P(L \cap S \not = \emptyset | S \in C) = \frac{|C|-1}{N}$
    \subitem If $|L| = 1$ then $P(L \cap S \not = \emptyset and S \in C) = \frac{|C|*(|C|-1)}{N*(N-1)}$
  \item Therefore there is at least a $\frac{|C|*(|C|-1)}{N*(N-1)}$ chance colluders will learn the secret at least one round early.
  \item If $N - C < M$ then also others will not learn the secret with that probability, since the coalition will stop participating. 
\end{itemize}



\subsection{Algorithm}

Differences from synchronous algorithm:

Rounds have H senders instead of all at once. Players will learn the secret H at a time, the round after they send. This is made possible by giving each player a different set of shares to reveal (i.e. they get different masks to apply), and making the masks apply on different rounds.

GRAPHIC

$H = 1$

$M = 3$

$N = 5$

Receivers

  Sender

. 1 2 3 4 5 1 2
  
2 + + + + + . x 5/5 pass

3 . + + + + + x 5/5 pass

4 . . + + + + x 4/5 pass

5 . . . + + + x 3/5 pass

1 . . . . + + x 2/5 fail



$H = 2$

$M = 3$

$N = 5$

Receivers

  Sender

. 61 23 45 61 23
  
2 .+ ++ ++ .. xx 5/5 pass

3 .+ ++ ++ .. xx 5/5 pass

4 .. .+ ++ ++ xx 5/5 pass

5 .. .+ ++ ++ xx 5/5 pass

6 .. .. .+ ++ xx 3/5 pass

1 .. .. .+ ++ xx 3/5 pass

\subsection{Analysis}



\section{limitations}
\section{future work}
\end{document}
